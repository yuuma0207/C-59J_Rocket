\documentclass[a4paper,11pt,titlepage,uplatex]{jsarticle}

% プリアンブルを外部ファイル化しておきました。中身はmacro.texで確認できます。

\usepackage[dvipdfmx]{graphicx,xcolor}% ドライバ指定
\usepackage[top=30truemm,bottom=30truemm,left=25truemm,right=25truemm]{geometry} % 余白設定

% 画像
\usepackage{here, subfig}
\usepackage{docmute} % ファイル分割用
\usepackage[cc]{titlepic}

% 数式関連
\usepackage{amsmath,amsfonts,amssymb,mathtools,amsthm}
\usepackage{bm} % ボールド体のベクトルを出力するときには\vb{a}ではなく\bm{a}としてください。\bmの方が綺麗に出力できる。
\usepackage{empheq} % 連立方程式をきれいに書いてくれる
\usepackage{physics} % 微分記号とか
\usepackage[separate-uncertainty]{siunitx} % SIUNITX

% 数式、図、表番号の変更
\makeatletter
\@addtoreset{equation}{section} % 章ごとに番号をリセット
\@addtoreset{figure}{section}
\@addtoreset{table}{section}
\def\theequation{\thesection.\arabic{equation}} % 章.何番目 と変更
\def\thefigure{\thesection.\arabic{figure}}
\def\thetable{\thesection.\arabic{table}}
\makeatother

% -------------------
% 定理環境付近
\usepackage{tcolorbox} % 色付きの囲み
\tcbuselibrary{breakable, skins, theorems}
\usepackage{ascmac} % 囲み \begin{itembox}ができる。

% ----------

\usepackage{enumitem} % enumium環境いじるために必要
\renewcommand{\labelenumi}{\theenumi.}
\renewcommand{\theenumi}{\Alph{enumi}}

% ------------ url関係
\usepackage{url}
\usepackage[dvipdfmx]{hyperref}
\hypersetup{
	 colorlinks=true,
	 citecolor=blue,
	 linkcolor=black,
	 urlcolor=blue
}
\usepackage{pxjahyper}
% ---------

% 表関連のパッケージ
\usepackage{booktabs}
\usepackage{multirow}
\usepackage{longtable}
\usepackage{arydshln}% 表で破線を使うため
\usepackage{multicol}
% longtableをusepackageする場合は順番が重要らしいです。longtableとarydshlnの順番逆にしたらエラーはく(コンパイルはできるが…)

\renewcommand{\labelitemii}{・}

\usepackage[greek, japanese]{babel}
\usepackage{teubner}	% 古代(古典)ギリシア語表記指定



% 大槻使用
\usepackage{color}
\newcommand{\red}[1]{\textcolor{red}{#1}}
\newcommand{\blue}[1]{\textcolor{blue}{#1}}
\usepackage{ulem}

% 能崎使用
%背景
\usepackage{wallpaper}

\begin{document}

\newpage

\section{プロジェクトのまとめ}
\subsection{サクセスクライテリアの達成状況について}
打上結果を基にして、サクセスクライテリアの達成状況を確認する。以下にサクセスクライテリアとその達成状況についてまとめた表を示す。
\begin{table}[H]
    \centering
    \caption{サクセスクライテリアとその達成状況}
    \begin{tabular}{cp{60mm}p{60mm}c} \toprule
         & \multicolumn{1}{c}{内容} & \multicolumn{1}{c}{判定条件} &達成状況\\ \midrule
    MIN  & 地上で制御プログラムを実行し、動翼の適切な動作を確認する。 & 動画とデータ解析により、意図した動翼の動作が実現していることを確認する。&達成 \\ \midrule
    FULL & ロール制御を成功させる。 & 搭載カメラの映像とデータ解析によって確認する。 &達成\\ \midrule
    ADV & 制御前に機体がロールをしていた場合、制御中は目標角からのロール角度を\SI{90}{\degree}以下にする。 & データ解析で内容を達成しているか確認する。 & 達成\\
    \bottomrule
    \end{tabular}
    \label{tab:success_criteria_2}
\end{table}
まず、地上試験において動翼の適切な動作の確認は打上前に行った。そのため、MINは達成できている。また、FULLについては機体搭載のカメラ映像と6軸センサにより、制御中の機体のロール角度が一定に保たれていることが確認された。そのため、FULLも達成できている。最後にADVについては、\ref{sc:data_doyoku}節で考察しているように非常に精度の良いロール制御ができており、明らかにクライテリアの内容は達成できている。
以上より、本プロジェクトのサクセスクライテリアはMIN、FULL、ADV全てを達成できたと結論付ける。

\subsection{今後の展望}
本プロジェクトの結果は打上前の期待を大幅に上回るものとなった。搭載していた全センサのデータを完全な形で取得することができ、団体初のロール制御も非常に精度よく行うことができた。
以上の結果を踏まえると、今後は更なる技術発展を目的としたピッチ・ヨー制御に挑戦することも十分に可能である。
また、今回のロール制御は機体のロール角を制御中一定に保つことを目的として行われたが、目標ロール角に収束後、新たな目標ロール角を設定して
その角度へ制御することに挑戦することも面白いと考えている。
また、二段式ロケットであるC-43Jでは機体の姿勢が不安定であったがために二段目の射出を行うことができなかった。
そのため、姿勢制御技術を発展させて二段式ロケットプロジェクトに再挑戦するといった取り組みも行えれば良いと考えている。

\section{謝辞}


\end{document}