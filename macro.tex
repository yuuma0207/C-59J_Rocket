\usepackage[dvipdfmx]{graphicx,xcolor}% ドライバ指定
\usepackage[top=30truemm,bottom=30truemm,left=25truemm,right=25truemm]{geometry} % 余白設定

% 画像
\usepackage{here, subfig}
\usepackage{docmute} % ファイル分割用
\usepackage[cc]{titlepic}

% 数式関連
\usepackage{amsmath,amsfonts,amssymb,mathtools,amsthm}
\usepackage{bm} % ボールド体のベクトルを出力するときには\vb{a}ではなく\bm{a}としてください。\bmの方が綺麗に出力できる。
\usepackage{empheq} % 連立方程式をきれいに書いてくれる
\usepackage{physics} % 微分記号とか
\usepackage[separate-uncertainty]{siunitx} % SIUNITX

% 数式、図、表番号の変更
\makeatletter
\@addtoreset{equation}{section} % 章ごとに番号をリセット
\@addtoreset{figure}{section}
\@addtoreset{table}{section}
\def\theequation{\thesection.\arabic{equation}} % 章.何番目 と変更
\def\thefigure{\thesection.\arabic{figure}}
\def\thetable{\thesection.\arabic{table}}
\makeatother

% -------------------
% 定理環境付近
\usepackage{tcolorbox} % 色付きの囲み
\tcbuselibrary{breakable, skins, theorems}
\usepackage{ascmac} % 囲み \begin{itembox}ができる。

% ----------

\usepackage{enumitem} % enumium環境いじるために必要
\renewcommand{\labelenumi}{\theenumi.}
\renewcommand{\theenumi}{\Alph{enumi}}

% ------------ url関係
\usepackage{url}
\usepackage[dvipdfmx]{hyperref}
\hypersetup{
	 colorlinks=true,
	 citecolor=blue,
	 linkcolor=black,
	 urlcolor=blue
}
\usepackage{pxjahyper}
% ---------

% 表関連のパッケージ
\usepackage{booktabs}
\usepackage{multirow}
\usepackage{longtable}
\usepackage{arydshln}% 表で破線を使うため
\usepackage{multicol}
% longtableをusepackageする場合は順番が重要らしいです。longtableとarydshlnの順番逆にしたらエラーはく(コンパイルはできるが…)

\renewcommand{\labelitemii}{・}

\usepackage[greek, japanese]{babel}
\usepackage{teubner}	% 古代(古典)ギリシア語表記指定



% 大槻使用
\usepackage{color}
\newcommand{\red}[1]{\textcolor{red}{#1}}
\newcommand{\blue}[1]{\textcolor{blue}{#1}}
\usepackage{ulem}

% 能崎使用
%背景
\usepackage{wallpaper}